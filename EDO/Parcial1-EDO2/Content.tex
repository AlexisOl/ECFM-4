 %%% EJERCICIO 1 
\begin{prob}
  Resolver $$x^2y'' + xy' + (x^2 - 1)y = 0$$  
\end{prob}
\begin{mdframed}

\begin{gather*}
   \frac{1}{x}, 1-\frac{1}{x^2}, x =0 \in A'\\
    y = \sum_{n=0}^{\infty} a_n x^{n+r}\\
    y' = \sum_{n=0}^{\infty} ({n+r})a_n x^{n+r-1}\\
    y'' = \sum_{n=0}^{\infty}  ({n+r})*({n+r-1}) a_n x^{n+r-2}\\
    \sum_{n=0}^{\infty}  ({n+r})*({n+r-1}) a_n x^{n+r}+ \sum_{n=0}^{\infty} ({n+r})a_n x^{n+r}+\sum_{n=0}^{\infty} a_n x^{n+r+2}-\sum_{n=0}^{\infty} a_n x^{n+r}=0\\
    m = n | m = n+2 \longrightarrow n = m-2\\
    \sum_{m=0}^{\infty}  ({m+r})*({m+r-1}) a_m x^{m+r}+ \sum_{m=0}^{\infty} ({m+r})a_m x^{m+r}+\sum_{m=2}^{\infty} a_{m-2} x^{m+r}-\sum_{m=0}^{\infty} a_m x^{m+r}=0\\
    m =0, 1\\
    (r*(r-1)a_0x^{r}+({r})a_0 x^{r}-a_0 x^{r})  +  (({1+r})*({r}) a_1 x^{1+r}+ ({1+r})a_1 x^{1+r}-a_1 x^{1+r})=0\\
    \text{y por independencia lineal}\\
\end{gather*}
  \begin{equation}
    \begin{cases}
        (r*(r-1)+({r})-1)=0\longrightarrow \pm1\\
        (({1+r})*({r})+ ({1+r})-1)=0\longrightarrow r_1 =0, r_2=-2\\
    \end{cases}
\end{equation}
Como $a_0 \neq 0$ entonces nos podemos dar la libertad de definir el siguiente factor $a_1$ como cero, asi siendo la serie y nuevamente por independencia lineal.
\begin{gather*}
       [({m+r})*({m+r-1})+ ({m+r}) -1]a_m+ a_{m-2} =0\\
       a_m=\frac{-a_{m-2}}{ [({m+r})*({m+r-1})+ ({m+r}) -1]}\\
       a_m=\frac{-a_{m-2}}{ [({m+r})^2 -1]}\\
       r_1 = 1\\
       a_m=\frac{-a_{m-2}}{ [({m+1})^2 -1]}\\
       m>=2\\
\end{gather*}
  \begin{equation}
    \begin{cases}
        a_2=\frac{-a_{0}}{8}\\
        a_3=0\\
        a_4=\frac{a_{0}}{192}\\
        a_5=0\\
        a_6=\frac{-a_{0}}{9216}\\
    \end{cases}
\end{equation}
\begin{gather*}
    y_1 = Ax(1-\frac{x^2}{8}+\frac{x^4}{192}-\frac{x^6}{9216}+\cdots)\\
    y_1 = Ax(1-\frac{x^2}{2^2(1+1)}+\frac{x^4}{2^5(1+1)(1+2)}-\cdots)\\
    y_1 = x\sum_{k=0}^{\infty}\frac{(-1)^k a_0 x^{2k}}{k!*4^k*(p+k)\cdot(p+1)}
\end{gather*}
este es la funcion de Bessel con x = 1, asi usamos la otra respuesta del punto, que es -1.
$$
    y_2 = x^{-1}\sum_{k=0}^{\infty}\frac{(-1)^k a_0 x^{2k}}{k!*4^k*(-p+k)\cdot(-p+1)}
$$
$$Y= y_1 +y_2 = J_1(x)+J_{-1}(x) \blacksquare$$
\end{mdframed}



%%% EJERCICIO 2
\begin{prob}
 La ecuación
$$x(1 - x)y'' + [\gamma - (\alpha + \beta + 1)x]y' - \alpha\beta y = 0$$
con $\alpha, \beta, \gamma$ constantes, se conoce como ecuación hipergeométrica o ecuación de Gauss.
\begin{itemize}
    \item {Muestre que x = 0 es un punto singular regular.}
    \item {
    Los valores de r en la serie de Frobenius son $r = 0$ y $r = 1 - \gamma$
    }
    \item {Muestre que
        $$a_{n+1}=\frac{(n+r+\alpha)(n+r+\beta)}{(n+r+1)(n+r+\gamma)}a_{n}, n =0,1,2,\cdots$$
    }
\end{itemize}
\end{prob}
    
    \begin{mdframed}
    \begin{itemize}
        \item {
             Primero para ver los puntos singulares regulares, debemos ver que las funciones P(x) y Q(x), Serán ambas analíticas en una vecindad otorgada por algún cambio que las vuelva analíticas.
        \begin{gather*}
         P(x)=\lim_{x\longrightarrow0} x\frac{ [\gamma - (\alpha + \beta + 1)x]}{x(1 - x)}= \gamma\\
         Q(x)= \lim_{x\longrightarrow0} x^2\frac{ - \alpha\beta}{x(1 - x)}=0\\
        \end{gather*}
        Así siendo cero un punto singular regular
        }
        \item {
            \begin{gather*}
                x(1-x)\sum_{n=0}^{\infty}  ({n+r})*({n+r-1}) a_n x^{n+r-2}+[\gamma - (\alpha + \beta + 1)x]\sum_{n=0}^{\infty} ({n+r})a_n x^{n+r-1}-\alpha\beta\sum_{n=0}^{\infty} a_n x^{n+r}\\
                \sum_{n=0}^{\infty}  ({n+r})*({n+r-1}) a_n x^{n+r-1}-\sum_{n=0}^{\infty}  ({n+r})*({n+r-1}) a_n x^{n+r}\\
                +\gamma\sum_{n=0}^{\infty} ({n+r})a_n x^{n+r-1}-(\alpha + \beta + 1)\sum_{n=0}^{\infty} ({n+r})a_n x^{n+r}-\alpha\beta\sum_{n=0}^{\infty} a_n x^{n+r}=0\\
                u = n-1\\
                \sum_{u=-1}^{\infty} ({u+r})*({u+r+1}) a_{u+1} x^{u+r}-\sum_{u=0}^{\infty}  ({u+r})*({u+r-1}) a_u x^{u+r}\\
                +\gamma\sum_{u=-1}^{\infty} ({u+1+r})a_{u+1} x^{u+r}-(\alpha + \beta + 1)\sum_{u=0}^{\infty} ({u+r})a_u x^{u+r}-\alpha\beta\sum_{u=0}^{\infty} a_u x^{u+r}=0\\
                \text{Por independencia lineal}\\
                a_0(r-1)(r)x^{r-1}+\gamma(r)a_0x^{r-1}=0\\
                (r-1)(r)+\gamma*r=0\\
            \end{gather*}
            \begin{equation}
                \begin{cases}
                   r_1=0\\
                   r_2=1-\gamma\\
                \end{cases}
            \end{equation}
        }
        \item {
            Nuevamente por independencia lineal
            \begin{gather*}
                (({u+r})*({u+r+1})+\gamma({u+r+1}))a_{u+1}+(-({u+r})*({u+r-1}) -(\alpha + \beta + 1)({u+r})-\alpha\beta)a_u =0\\
                (({u+r})*({u+r+1})+\gamma({u+r+1}))a_{u+1} =(({u+r})*({u+r-1}) +(\alpha + \beta + 1)({u+r})+\alpha\beta)a_u\\
                a_{u+1} =\frac{(({u+r})*({u+r-1}) +(\alpha + \beta + 1)({u+r})+\alpha\beta)}{(({u+r})*({u+r+1})+\gamma({u+r+1}))}a_u\\
                a_{u+1} =\frac{(({u+r+\alpha})+({u+r+\beta}))}{(({u+r+1})+({u+r+\gamma}))}a_u\\
            \end{gather*}
            esta serie desde u = 0, con $u\in I | I =\{0,1,\cdots\} \blacksquare $
        }
    \end{itemize}
    \end{mdframed}


%%% EJERCICIO 3
\begin{prob}
Denotemos a la solución $y_1$ para $r = 0$ y $a_0 = 1$ de la ecuación hipergeométrica como $F (\alpha, \beta, \gamma, x)$
\begin{itemize}
    \item {Muestre que 
    $$F (1, \beta, \beta, x)= \frac{1}{1-x} $$
    }
    \item {Muestre que 
    $$F (\alpha, \beta, \beta, x)=\frac{1}{(1-x)^{\alpha}}  $$
    }
    \item {Muestre que 
    $$ xF (\frac{1}{2}, \frac{1}{2}, \frac{3}{2}, x^2)={\arcsin(x)}$$
    }
\end{itemize}

\end{prob}
\begin{mdframed}
    \begin{itemize}
        \item {
            bueno en este caso tenemos que
            \begin{gather*}
                F_a = a_{u+1}= \frac{(({u+1}))}{(({u+1}))}a_u\\
            \end{gather*}
            donde debe de converger la susecion por $\frac{(n)}{(k)}$ asi generando
            $$\frac{1}{k}$$, sea este valor maximo, con lo cual 
            $$\sum_{n=0}^{\infty}\frac{(1)_n}{n!}x^n = \frac{1}{1-x}$$, esto debera ser igual a la serie geometria. Esto tambien se aprecia bien con la extension en serie.
            $$F = 1+\frac{1*\beta}{\beta*1}x++\frac{1*\beta*(1+1)*(1+\beta)}{\beta*(1+\beta)*1*2}x^2+\cdots$$
            $$F = 1+x+x^2+...= \frac{1}{1-x}$$
        }
        \item {
            Con lo anterior podemos ver que si ahora en vez de tener $\alpha =1$, empleamos el propio valor de alpha, nos dara casi lo mismo, solo que en vez de ser esa serie geometrica, esta sera
            \begin{gather*}
                F = 1+\frac{\alpha*\beta}{\beta*1}x+\frac{\alpha*\beta*(\alpha+1)*(1+\beta)}{\beta*(1+\beta)*1*2}x^2+\cdots\\
                F = 1+\frac{\alpha}{1}x+\frac{\alpha*(\alpha+1)}{1*2}x^2+\cdots\\
            \end{gather*}
            Esta forma es de $(a+b)^n =a^n+n*a^{n-1}b+\cdots n*a*b^{n-1}+b^n$, solo que con los valores estipulados asi, siendo
            $$F = \frac{1}{(1-x)^{\alpha}}$$
        }
        \item {
        Tratemos de llevarlo con pura funcion Gamma
            \begin{gather*}
                x\sum \frac{(\frac{1}{2})_n^2}{(\frac{3}{2})}*\frac{x^{2n}}{n!}\\
                \sum \frac{\Gamma(\frac{1}{2}+n)^2\Gamma(\frac{3}{2})}{\Gamma(\frac{1}{2})^2\Gamma(\frac{3}{2})+n}*\frac{x^{2n+1}}{n!}\\
                \sum \frac{\Gamma(\frac{1}{2}+n)^2}{(2n+1)\sqrt{\pi}}*\frac{x^{2n+1}}{n!}\\
                \sum \frac{(2n)!}{(2^{2n}n!)}*\sqrt{\pi}*\frac{1}{{2n+1}\sqrt{\pi}}*\frac{x^{2n+1}}{n!}\\
                \sum \frac{(2n)!}{2^{2n}(n!)^2}*\frac{x^{2n+1}}{2n+1} = \arcsin(x) \blacksquare
            \end{gather*}
        }
    \end{itemize}
\end{mdframed}
