 %%% EJERCICIO 1 
\begin{prob}
  Dado el Sistema Lineal $$ \diff{x}{t}= x -\frac{2}{3} y + t $$  
   $$ \diff{y}{t}= \frac{4}{3}x  - y +1$$  
  \begin{enumerate}
      \item {
      Muestre que la EDO satisface las condiciones del Teorema de Existencia y Unicidad. Utilice
las iteraciones de Picard para encontrar una aproximación a la solución
      }
      \item {
      Resuelva el sistema con el método de matrices.
      }
  \end{enumerate}

\end{prob}
\begin{mdframed}

\begin{enumerate}
    \item {
        Bueno en este caso vamos a proboar con existencia unicidad, para ello vamos a definir unas cosas, la forma de ver este sistema sera: $f(T, t) = A(t)T +g(t)$, gracias a lo siguiente:
        \begin{gather*}
        \overline{T}'= 
            \begin{pmatrix}
                {x}' \\
                {y}' 
            \end{pmatrix} =  
            \underbrace{\begin{pmatrix}
                1 & -\frac{2}{3}\\
                \frac{4}{3} & -1 
            \end{pmatrix}}_{A}*
            \underbrace{\begin{pmatrix}
                x \\
                y 
            \end{pmatrix}}_{T} +\underbrace{\begin{pmatrix}
                t \\
                1 
                \end{pmatrix}}_{g}\\
        \end{gather*}
        Primero que sea de lipschitz.
        \begin{gather*}
            ||f(X, t)-f(Y, t)||=|| A(t)X +g(t) -A(t)Y - g(t)||\\
            ||  A(t)*(x-y)|| \leq |A(t)| ||x-y|| = A||x-y||
        \end{gather*}
        Ahora con eso como es de lipschitz(gracias a Cauchy-schwarz), es uniformemente continua y debera se continua.
        BUeno y con Picard
        \begin{gather*}
            y_{1,0}=x_0;
            y_{2,0}=y_0\\
            y_{1,1}=x_0+\int_{0}^{\tau}y_{1,0}(n)-\frac{2}{3}y_{2,0}(n)+tdn\\
            y_{2,1}=y_0+\int_{0}^{t}\frac{4}{3}y_{1,0}(n)-y_{2,0}(n)+1dn\\
            y_{1,1}= x_0+x_0\tau-\frac{2}{3}y_0\tau+t\tau\\
            y_{2,1}= y_0+\frac{4}{3}x_0\tau-y_0\tau+\tau\\
            y_{1,2}=x_0+\int_{0}^{t}y_{1,1}(n)-\frac{2}{3}y_{2,1}(n)+ndn\\
            y_{2,2}=y_0+\int_{0}^{t}\frac{4}{3}y_{1,1}(n)-y_{2,1}(n)+1dn\\
            y_{1,2}=x_0+\int_{0}^{t}x_0-\frac{2}{3}y_0+\frac{1}{9}x_0n+tn-\frac{2}{3}n dn\\
            y_{1,2}=x_0+x_0\tau-\frac{2}{3}y_0\tau+\frac{1}{18}x_0\tau^2+\frac{t \tau^2}{2}-\frac{\tau^2}{3}\\
            y_{2,2}=y_0+\int_{0}^{t}\frac{4}{3}x_0-y_0+\frac{1}{9}y_0+\frac{4}{3}tn-ndn\\
            y_{2,2}=y_0+\frac{4}{3}x_0\tau-y_0\tau+\frac{1}{18}y_0\tau^2+\frac{2}{3}t\tau^2-\frac{\tau^2}{2}\\
        \end{gather*}
        Donde mi funcion debe de tender a $y_{1, n}$
    }
       \item {
       Bueno al ser no homogenea, entonces hacemos variacion de parametros, en este caso la forma matricial seria, donde X es la matriz de variables diferenciadas:
        \begin{gather*}
            X = 
                \begin{pmatrix}
                1 & -\frac{2}{3}\\
                \frac{4}{3} & -1 
                \end{pmatrix}
            +  \begin{pmatrix}
                t \\
                1 
                \end{pmatrix}\\
            Det(A-\lambda I) = 
                \begin{pmatrix}
                1 -\lambda& -\frac{2}{3}\\
                \frac{4}{3} & -1-\lambda 
                \end{pmatrix}\\
            \text{con el polinomio caracteristico:}
            P_A(x)= x^2 -\frac{1}{9}
        \end{gather*}
        \begin{equation}
                \begin{cases}
                   \lambda_1 = \frac{1}{3}\\
                   \lambda_2 = -\frac{1}{3}\\
                \end{cases}
        \end{equation}
        Ambos con multiplicidad Algebraica de 1.
            \begin{itemize}
                \item {
                $\lambda_1$
                \begin{gather*}
                    \begin{pmatrix}
                    1 -\frac{1}{3}& -\frac{2}{3}\\
                    \frac{4}{3} & -1-\frac{1}{3} 
                    \end{pmatrix}
                    \longrightarrow
                    \begin{pmatrix}
                    \frac{2}{3}& -\frac{2}{3}\\
                    \frac{4}{3} & -\frac{4}{3} 
                    \end{pmatrix}
                    \\
                    x=y \rightarrow span\{\begin{pmatrix}
                1 \\
                1 
                \end{pmatrix}\}\\
                Dim(E(\lambda_1)) = 1
                \end{gather*}
                }
                   \item {
                $\lambda_2$
                \begin{gather*}
                    \begin{pmatrix}
                    1 +\frac{1}{3}& -\frac{2}{3}\\
                    \frac{4}{3} & -1+\frac{1}{3} 
                    \end{pmatrix}
                    \longrightarrow
                    \begin{pmatrix}
                    \frac{4}{3}& -\frac{2}{3}\\
                    \frac{4}{3} & -\frac{2}{3} 
                    \end{pmatrix}
                    \\
                    2x=y \rightarrow span\{\begin{pmatrix}
                1 \\
                2 
                \end{pmatrix}\}\\
                Dim(E(\lambda_2)) = 1
                \end{gather*}
                }
            \end{itemize}
    }
    Gracias a que ambos tienen misma cantidad de ambas multiplicidades (algebraica y geometrica) con esto solo tendremos un bloque de Jordan (el general vaya).
    \begin{gather*}
        y = C_1 \begin{pmatrix}
                1 \\
                1 
                \end{pmatrix}
            e^{\frac{t}{3}}
            + C_2 \begin{pmatrix}
                1 \\
                2 
                \end{pmatrix}
            e^{\frac{-t}{3}}\\
                \underbrace{\begin{pmatrix}
                 e^{\frac{t}{3}}&  e^{\frac{-t}{3}}\\
                 e^{\frac{t}{3}} & 2 e^{\frac{-t}{3}} 
                \end{pmatrix}}_{\Phi}
                \begin{pmatrix}
                C_1 \\
                C_2 
                \end{pmatrix}\\
        |\Phi| = 1\\
        adj(\Phi)^t=
                \begin{pmatrix}
                 2e^{\frac{-t}{3}}&  -e^{\frac{-t}{3}}\\
                 -e^{\frac{t}{3}} & e^{\frac{t}{3}} 
                \end{pmatrix}\\
        \Phi^{-1}=|\Phi|* adj(\Phi)^t=  \begin{pmatrix}
                                         2e^{\frac{-t}{3}}&  -e^{\frac{-t}{3}}\\
                                         -e^{\frac{t}{3}} & e^{\frac{t}{3}} 
                                        \end{pmatrix}\\
        \Phi^{-1}*\overline{F} = \begin{pmatrix}
                            2e^{\frac{-t}{3}}&  -e^{\frac{-t}{3}}\\
                            -e^{\frac{t}{3}} & e^{\frac{t}{3}} 
                            \end{pmatrix}*\begin{pmatrix}
                            t \\
                            1 
                            \end{pmatrix}\\
        \int{\begin{pmatrix}
            2te^{\frac{-t}{3}} -e^{\frac{-t}{3}}\\
            -te^{\frac{t}{3}} + e^{\frac{t}{3}} 
            \end{pmatrix} dt}\\
        \begin{pmatrix}
            -15e^{\frac{-t}{3}} -6te^{\frac{-t}{3}}\\
            12e^{\frac{t}{3}} - 3te^{\frac{t}{3}} 
        \end{pmatrix}\\
        \Phi*\int{\Phi^{-1}*\overline{F}}\\
        \begin{pmatrix}
                 e^{\frac{t}{3}}&  e^{\frac{-t}{3}}\\
                 e^{\frac{t}{3}} & 2 e^{\frac{-t}{3}} 
        \end{pmatrix} 
        * \begin{pmatrix}
            -15e^{\frac{-t}{3}} -6te^{\frac{-t}{3}}\\
            12e^{\frac{t}{3}} - 3te^{\frac{t}{3}} 
        \end{pmatrix}\\
        \begin{pmatrix}
            -3-9t\\
            9-12t 
        \end{pmatrix}\\
        \begin{pmatrix}
                -3 \\
                9 
        \end{pmatrix}+\begin{pmatrix}
                -9 \\
                -12 
        \end{pmatrix}t\\
         y = C_1 \begin{pmatrix}
                1 \\
                1 
                \end{pmatrix}
            e^{\frac{t}{3}}
            + C_2 \begin{pmatrix}
                1 \\
                2 
                \end{pmatrix} +\begin{pmatrix}
                -3 \\
                9 
        \end{pmatrix}+\begin{pmatrix}
                -9 \\
                -12 
        \end{pmatrix}t \text{\space} \blacksquare
    \end{gather*}
\end{enumerate}
\end{mdframed}



%%% EJERCICIO 2
\begin{prob}
Considere dos puntos A1 y A2, r1 la distancia del origen al punto A1, r2 la distancia del
origen al punto A2 , ambas rectas forman un ángulo $\theta$ con vértice en el origen. Si r es la
distancia de A1 a A2 , muestre que
$$\frac{1}{r} = \frac{1}{r_2}\sum_{n=0}^\infty P_n(\cos(\theta))\left(\frac{r_1}{r_2}\right)^n$$
\end{prob}
    
    \begin{mdframed}
    \begin{center}
         \begin{tikzpicture}
        \draw[lightgray] (-1, -2) grid (7, 5);
        \coordinate[label=left:$O$] (A) at (1,0);
        \coordinate[label=left:$A_1$] (B) at (0,4);
        \coordinate[label=above:$A_2$] (C) at (5,2);
        \draw[thick] (A)--(B) node[midway, above left] {$r_1$}-- (C)node[midway, below] {$r$} --cycle;
        \draw (A) -- (C) node[midway, below] {$r_2$};
        \draw pic[draw, angle eccentricity=1.2, angle radius=1cm, "$\theta$"] {angle = C--A--B};
    \end{tikzpicture}
    \end{center}
    AHora con esto usemnos la funcion generatriz
    \begin{gather*}
        \sum_{n=0}^{\infty}P_n(x)t^n = \frac{1}{\sqrt{1-2xt+t^2}}\\
        \frac{1}{r_2}\sum_{n=0}^{\infty}P_n(\cos(\theta))\left(\frac{r_1}{r_2}\right)^n =\frac{1}{r_2}* \frac{1}{\sqrt{1-2\cos(\theta)\left(\frac{r_1}{r_2}\right)+\left(\frac{r_1}{r_2}\right)^2}}\\
        \frac{1}{r_2}*\frac{1}{\sqrt{r_2^2-2\cos(\theta)r_1*r_2+\left({r_1}\right)^2}}*r_2\\
        \frac{1}{\sqrt{r_2^2-2\cos(\theta)r_1*r_2+\left({r_1}\right)^2}}\\
        \text{por ley de cosenos}\\
        r^2=r_1^2+r_2^2-2\cos(\theta)r_1r_2\\
        \frac{1}{\sqrt{r^2}}\longrightarrow\frac{1}{||r||_1}\\
        \text{por geometria todo es positivo}\\
        \frac{1}{\sqrt{r^2}}\longrightarrow\frac{1}{r} \blacksquare
    \end{gather*}

    \end{mdframed}


%%% EJERCICIO 3
\begin{prob}
Utilizando series, resuelva la EDO
$$y'' + (e^x - 9)y = 0$$
\end{prob}
\begin{mdframed}
    Para este caso, pues acoplare la serie de taylor en base a todos los valores que tengo de las series, antes de eso hay que percatarse que como no tiene puntos singulares, no se usa Frobenius, simplemente serie de potencias. Y recordando que Euler es:
    $$e^x = \sum_{n=0}^{\infty} \frac{x^n}{n!}$$
    AHora si con esto podemos comenzar
    \begin{gather*}
        y = \sum_{n=0}^{\infty}   a_n x^{n}\\
        y' = \sum_{n=0}^{\infty}  ({n}) a_n x^{n-1}\\
        y'' = \sum_{n=0}^{\infty}  ({n})*({n-1}) a_n x^{n-2}\\
         \sum_{n=0}^{\infty}  ({n})*({n-1}) a_n x^{n-2} + e^x \sum_{n=0}^{\infty} a_n x^{n}-9\sum_{n=0}^{\infty} a_n x^{n}\\
        \underbrace{(2a_2+6a_3x+12a_4x^2+\dots)}_{y''}+\underbrace{(1+x+\frac{x^2}{2!}+\frac{x^3}{3!}+\dots)}_{e^x}*\underbrace{(a_0+a_1x+a_2x^2+\dots)}_{y}-9\underbrace{(a_0+a_1x+a_2x^2+\dots)}_{y}=0\\
        \underline{\begin{matrix}
        a_0 & a_1x & a_2x & \cdots\\
         & a_0x & a_1x^2 & \cdots \\
         &     & \frac{a_0}{2!}x^2 & \cdots \\
         &     &  & \ddots \\
        \end{matrix}}\\
        a_0+(a_1+a_0)x+(a_2+a_1+\frac{a_0}{2!})x^2+\dots+\left(\frac{a_n}{0!}+\frac{a_{n-1}}{1!}+\frac{a_{n-2}}{2!}+\frac{a_{0}}{{n}!} \right)x^n\\
        2a_2+a_0+(6a_3+a_1+a_0)x+(12a_4+a_2+a_1+\frac{a_0}{2!})x^2+\dots - 9a_0-9a_1x-\dots\\
        2a_2-8a_0+(6a_3-8a_1+a_0)x+(12a_4-8a_2+a_1+\frac{a_0}{2!})x^2+\dots =0\\
        \text{al final se puede hacer todo eso por Independencia Lineal :)}
    \end{gather*}
          \begin{equation}
                \begin{cases}
                   a_2=4a_0\\
                   a_3=\frac{8a_1-a_0}{6}\\
                   a_4=\frac{63}{24}a_0-\frac{a_1}{12}\\
                \end{cases}
            \end{equation}
    \begin{align*}
        y &= a_0+a_1x+a_2x^2+\dots\\
        &=a_0+a_1x+4a_0x^2+\frac{4}{3}a_1x^3-\frac{1}{6}a_0x^3+\frac{21}{8}a_0x^4-\frac{a_1}{12}x^4+\dots\\
        &= (1+4x^2-\frac{1}{6}x^3+\frac{21}{8}x^4)a_0+(x+\frac{4}{3}x^3-\frac{1}{12}x^4)a_1 \blacksquare
        \end{align*}
\end{mdframed}
%%% EJERCICIO 4
\begin{prob}
 Muestre que los polinomios de Legendre $P_n(x)$ satisfacen la recurrencia
$$xP'_n(x)-P'_{n-1}(x) = nP_n(x)$$
\end{prob}
\begin{mdframed}
    Podemos usar la funcion generatriz de Legendre para esto, esto porque aqui si la puedo derivar mejor, a comparacion de usar Rodrigues(aunque imagino que jala tambien)
    \begin{gather*}
        \sum_{n=0}^{\infty}P_n(x)t^n = \frac{1}{\sqrt{1-2xt+t^2}}\\
        \diff{}{x}P_n(x)=\frac{t}{\sqrt{{(1-2xt+t^2)}^3}}\\
        P'_{n-1}(x)=\sum_{n=0}^{\infty}P'_{n-1}(x)t^n \longrightarrow
        \sum_{n=0}^{\infty}P'_n(x)t^{n+1}\\
        P'_{n-1}(x)=t\sum_{n=0}^{\infty}P'_{n}(x)t^n \longrightarrow tP'n\\
        xP'_n(x)-P'_{n-1}(x) =  xP'_n(x)-tP'_{n}(x)\longrightarrow(x-t)P'n(x)\\
        (x-t)\frac{t}{\sqrt{{(1-2xt+t^2)}^3}}\\
        \diff{}{t}P_n=\sum_{n=0}^{\infty}nt^{n-1}P_n(x)\\
        \diff{}{t}P_n= \frac{1}{\sqrt{(1-2xt+t^2)^3}}*\frac{2x-2t}{2}\\
        \diff{}{t}P_n= \frac{x-t}{\sqrt{(1-2xt+t^2)^3}}\\
        \diff{}{t}P_n=\sum_{n=0}^{\infty}nt^{n-1}P_n(x)\\
        (x-t)= \sum_{n=0}^{\infty}nt^{n-1}P_n(x)*\sqrt{(1-2xt+t^2)^3}\\
        (x-t)\frac{t}{\sqrt{{(1-2xt+t^2)}^3}}\longrightarrow \sum_{n=0}^{\infty}nt^{n-1}P_n(x)*\sqrt{(1-2xt+t^2)^3}*\frac{t}{\sqrt{{(1-2xt+t^2)}^3}}\\
        t\sum_{n=0}^{\infty}nt^{n-1}P_n(x) \longrightarrow \sum_{n=0}^{\infty}nt^{n}P_n(x) = nP_n(x) \blacksquare
    \end{gather*}
\end{mdframed}
