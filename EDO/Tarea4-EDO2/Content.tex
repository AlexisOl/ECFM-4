

%%% EJERCICIO 1
\begin{prob}
Resolver $$x^2y'' +x(x+\frac{1}{2})y'+xy = 0$$

\end{prob}
\begin{mdframed}
    Comencemos a ver los puntos singulares, para ver si toca Frobenius.
    \begin{gather*}
        y'' +(1+\frac{1}{2x})y'+\frac{1}{x}y = 0\\
        \text{Donde los puntos singulares serán:}
        \frac{1}{2x}, \frac{1}{x} | x = 0 \in A'\\
         y = \sum_{n=0}^{\infty} a_n x^{n+r}\\
         y'= \sum_{n=0}^{\infty} ({n+r})a_n x^{n+r-1}\\
         y'' = \sum_{n=0}^{\infty}  ({n+r})*({n+r-1}) a_n x^{n+r-2}\\
        \sum_{n=0}^{\infty}  ({n+r})*({n+r-1}) a_n x^{n+r}+\sum_{n=0}^{\infty} ({n+r})a_n x^{n+r+1}+\frac{1}{2}\sum_{n=0}^{\infty} ({n+r})a_n x^{n+r}+\sum_{n=0}^{\infty} a_n x^{n+r+1}=0\\
        m = n+1, n=m-1 \vert m=n\\
        \sum_{m=0}^{\infty}  ({m+r})*({m+r-1}) a_m x^{m+r}+\sum_{m=1}^{\infty} ({m-1+r})a_{m-1} x^{m+r}+\frac{1}{2}\sum_{m=0}^{\infty} ({m+r})a_m x^{m+r}+\sum_{m=1}^{\infty} a_{m-1} x^{m+r}=0\\
        m = 1\\
        \sum_{m=1}^{\infty}  ({m+r})*({m+r-1}) a_m x^{m+r}+\sum_{m=1}^{\infty} ({m-1+r})a_{m-1} x^{m+r}+\frac{1}{2}\sum_{m=1}^{\infty} ({m+r})a_m x^{m+r}+\sum_{m=1}^{\infty} a_{m-1} x^{m+r}\\+
        ({r})*({r-1}) a_0 x^{r}+\frac{1}{2}({r})a_0 x^{r}=0\\
        \sum_{m=1}^{\infty}  [[({m+r})*({m+r-1})+\frac{1}{2}({m+r}) ]a_m+ [({m-1+r}) + 1]a_{m-1}]x^{m+r}+({r})*({r-1}) a_0 x^{r}+\frac{1}{2}({r})a_0 x^{r}=0\\
        \text{por independencia lineal}\\
        r^2 -\frac{1}{2}({r})=0 \Longrightarrow r(r-\frac{1}{2})=0\\
    \end{gather*}
    \begin{equation}
        \begin{cases}
            r_1=0\\
            r_2=\frac{1}{2}\\
        \end{cases}
    \end{equation}
    \begin{gather*}
        a_m=- \frac{[({m-1+r}) + 1]a_{m-1}}{[({m+r})*({m+r-1})+\frac{1}{2}({m+r}) ]}\\
        r_1 =0\\
        a_m=- \frac{[m]a_{m-1}}{[({m})*({m-1})+\frac{1}{2}({m})]}\\
        m >=1\\
\end{gather*}
  \begin{equation}
    \begin{cases}
        a_1=-{2a_0}\\
        a_2=\frac{4a_0}{3}\\
        a_3=-\frac{8a_0}{15}\\
    \end{cases}
\end{equation}
\begin{gather*}
    r_2 = \frac{1}{2}\\
    a_m=- \frac{[({m+\frac{1}{2}})]a_{m-1}}{[({m+\frac{1}{2}})*({m-\frac{1}{2}})+\frac{1}{2}({m+\frac{1}{2}}) ]}\\
    m >=1\\
\end{gather*}
  \begin{equation}
    \begin{cases}
        a_1=-{a_0}\\
        a_2=\frac{a_0}{2}\\
        a_3=-\frac{a_0}{6}\\
    \end{cases}
\end{equation}

Así siendo la solución
$$y = A(1-2x+\frac{4x^2}{3}-\frac{8x^3}{15}+\cdots)+\sqrt{x}B(1-x+\frac{x^2}{2}-\frac{x^3}{6}+\cdots) \blacksquare$$
\end{mdframed}

%%% EJERCICIO 2
\begin{prob}
        Resolver $$2x^2y''+3xy'-(2x-1)y=0$$
sucesivas.
\end{prob}
\begin{mdframed}
    Comencemos a ver los puntos singulares, para ver si toca Frobenius.
    \begin{gather*}
         2x^2y''+3xy'-(2x-1)y=0\\
        \text{Donde los puntos singulares serán:}\\
        \frac{2}{3x}, \frac{1}{x}-\frac{1}{2x^2} | x = 0 \in A'\\
         y = \sum_{n=0}^{\infty} a_n x^{n+r}\\
         y'= \sum_{n=0}^{\infty} ({n+r})a_n x^{n+r-1}\\
         y'' = \sum_{n=0}^{\infty}  ({n+r})*({n+r-1}) a_n x^{n+r-2}\\
         2\sum_{n=0}^{\infty}  ({n+r})*({n+r-1}) a_n x^{n+r}+3\sum_{n=0}^{\infty} ({n+r})a_n x^{n+r}+2\sum_{n=0}^{\infty} a_n x^{n+r+1}-\sum_{n=0}^{\infty} a_n x^{n+r}=0\\
         m = n+1, n=m-1 \vert m=n\\
         2\sum_{m=0}^{\infty}  ({m+r})*({m+r-1}) a_m x^{m+r}+3\sum_{m=0}^{\infty} ({m+r})a_m x^{m+r}+2\sum_{m=1}^{\infty} a_{m-1} x^{m+r}-\sum_{m=0}^{\infty} a_m x^{m+r}=0\\
         \text{por ser Linealmente independiente, queda como:}\\
         2r^2+r-1=0\\
    \end{gather*}
        \begin{equation}
        \begin{cases}
            r_1=-1\\
            r_2=\frac{1}{2}\\
        \end{cases}
    \end{equation}
    \begin{gather*}
        \sum_{m=1}^{\infty}[(2({m+r})*({m+r-1}) +3({m+r})-1)a_m+2a_{m-1}] x^{m+r}=0\\
        a_m=\frac{-2a_{m-1}}{[(({m+r})*({2m+2r+1}) -1)]}\\
        r_1 = -1\\
        a_m=\frac{-2a_{m-1}}{[(({m-1})*({2m-1}) -1)]}\\
        n>=1\\
    \end{gather*}
    \begin{equation}
    \begin{cases}
        a_2=2{a_0}\\
        a_2=-2{a_0}\\
        a_3=\frac{4a_0}{9}\\
    \end{cases}
\end{equation}
    \begin{gather*}
        r_1 = \frac{1}{2}\\
        a_m=\frac{-2a_{m-1}}{[(({m+\frac{1}{2}})*({2m+2}) -1)]}\\
        n>=1\\
    \end{gather*}
    \begin{equation}
    \begin{cases}
        a_1=-2\frac{a_0}{5}\\
        a_2=2\frac{a_0}{35}\\
    \end{cases}
\end{equation}
Así siendo la solución
$$y = A*x^{-1}(1-2x-2x^2+\frac{4x^3}{9}+\cdots)+\sqrt{x}B(1-\frac{2x^2}{5}+\frac{2x^2}{35}+\cdots) \blacksquare$$
\end{mdframed}

%% EJERCICIO 3
\begin{prob}
    Resolver $$2xy''+(x+1)y'+3y=0$$
\end{prob}
\begin{mdframed}
  \begin{gather*}
      2xy''+(x+1)y'+3y=0\\
      \text{Donde los puntos singulares serán:}\\
    \frac{1}{2}+\frac{1}{2x}, \frac{3}{2x}| x = 0 \in A'\\
  \end{gather*}
  donde $P(x)$ y $Q(x)$, no son analíticas en $x_0 =0 \in A'$ así que usaremos Frobenius.
  \begin{gather*}
      y = \sum_{n=0}^{\infty} a_n x^{n+r}\\
      y' = \sum_{n=0}^{\infty} ({n+r})a_n x^{n+r-1}\\
      y'' = \sum_{n=0}^{\infty}  ({n+r})*({n+r-1}) a_n x^{n+r-2}\\
      2 \sum_{n=0}^{\infty}  ({n+r})*({n+r-1}) a_n x^{n+r-1}+\sum_{n=0}^{\infty} ({n+r})a_n x^{n+r}+\sum_{n=0}^{\infty} ({n+r})a_n x^{n+r-1}+3\sum_{n=0}^{\infty} a_n x^{n+r}=0\\
      m = n-1, n=m+1 \vert m=n\\
      2 \sum_{m=-1}^{\infty}  ({m+1+r})*({m+r}) a_{m+1} x^{m+r}+\sum_{m=0}^{\infty} ({m+r})a_m x^{m+r}+\sum_{m=-1}^{\infty} ({m+1+r})a_{m+1} x^{m+r}+3\sum_{m=0}^{\infty} a_m x^{m+r}=0\\
      m = 0\\
      \text{al ser linealmente independientes las soluciones de series}\\     
      2({r})*({r-1}) a_{-1} x^{r-1} +({r})a_{-1} x^{r-1}\longrightarrow r_1=0, r_2=\frac{1}{2}\\ 
      [ 2[({m+1+r})*({m+r}) +({m+1+r})]a_{m+1}+ [({m+r})+3]a_m ]=0\\
      a_{m+1}=-\frac{[({m+r})+3]a_m }{[2({m+1+r})*({m+r}) +({m+1+r})]}\\
      a_{m+1}=-\frac{[({m+3})]a_m }{[2({m})*({m+1}) +({m+1})]}\\
  \end{gather*}
  \begin{equation}
    \begin{cases}
        a_2=-3a_1\\
        a_3={2a_1}\\
        a_4=-\frac{2a_1}{3}\\
    \end{cases}
\end{equation}
\begin{gather*}
      a_{m+1}=-\frac{[({m+\frac{7}{2}})]a_m }{[2({m+\frac{3}{2}})*({m+\frac{1}{2}}) +({m+\frac{3}{2}})]}\\
\end{gather*}
  \begin{equation}
    \begin{cases}
        a_2=-\frac{7a_1}{6}\\
        a_3=\frac{21a_0}{40}\\
    \end{cases}
\end{equation}
Asi siendo la solucion
$$y = A(1+3x+2x^2-\frac{2x^3}{3}+\cdots)+x^{\frac{1}{2}}B(1-\frac{7x}{6}+\frac{21x^2}{40}+\cdots) \blacksquare$$
\end{mdframed}