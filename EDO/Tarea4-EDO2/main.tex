\documentclass[spanish,11ptm letterpage]{article}
\usepackage[letterpaper]{geometry} 
\geometry{verbose,tmargin=2.5cm,bmargin=2.5cm,lmargin=2cm,rmargin=2cm}
\usepackage{amsmath,amsthm,amssymb} 
\usepackage[utf8]{inputenc} 


\setlength{\jot}{12pt}

% Permitir la división de ecuaciones en varias páginas
\allowdisplaybreaks


\usepackage{latexsym}
\usepackage{amsfonts}
\usepackage{mathtools}
\usepackage[thinc]{esdiff}

\usepackage[utf8]{inputenc}
\usepackage{graphics}
\usepackage[spanish, es-noshorthands]{babel}  
%% ALGUNAS COSAS DE TABLAS
\usepackage{tabulary}
\usepackage{multirow}
\usepackage{multicol} %varias columnas
\usepackage{array}
\usepackage{graphicx} %graficas e imagenes



\title{Tarea 4}
\author{Alexis Ovalle}
\date{January 2023}

%%%%%%%%% COMANDOS NUEVOS
\usepackage{pgf,tikz,pgfplots}

\decimalpoint %cambia las comas por puntos decimal
%sangrias
\parindent =0cm 

\usepackage[framemethod=TikZ]{mdframed}%Entornos talegas

\usepackage{enumerate} %enumeraciones
\usepackage{mathrsfs} %letras chingonas (transformada de laplace)
\usepackage{subfigure} %varias figuras seguidas
\usepackage[square,numbers]{natbib} %bibliografias
\usepackage[nottoc]{tocbibind}
\bibliographystyle{plainnat}
\usetikzlibrary{arrows, babel, calc}
\usepackage{tabulary}
\usepackage{multirow} %ocupar varias filas en una tabla
\usepackage{fancybox} %recuadros talegas
\usepackage{float} %ubicar graficas
\usepackage{color}
\usepackage{comment}
\usepackage{stackrel}
\usepackage{calligra}
\usepackage{cite}
\usepackage{circuitikz} % crear circuitos
\usepackage{listings} % permite el ingreso de codigo
\usepackage{enumerate}
\usepackage{enumitem}
% NEW PACKAGES
\usepackage{makeidx}
\usepackage{authblk} % para la manipulación de autores y afiliación
\usepackage{booktabs}
\usepackage{colortbl}
\usepackage{bbold}
\usepackage{dsfont}
\usepackage{tensor}
\usepackage{colortbl}
\usepackage{amsbsy}
\usepackage[draft,inline,nomargin]{fixme} \fxsetup{theme=color}
\usepackage{tikz}
\usetikzlibrary{trees}
\usepackage{verbatim}

%FECHAS
\usepackage{advdate}
% Set the overall layout of the tree
\tikzstyle{level 1}=[level distance=3.5cm, sibling distance=3.5cm]
\tikzstyle{level 2}=[level distance=3.5cm, sibling distance=2cm]

% Define styles for bags and leafs
\tikzstyle{bag} = [text width=4em, text centered]
\tikzstyle{end} = [circle, minimum width=3pt,fill, inner sep=0pt]
%This defines my comments
\definecolor{mycolor}{RGB}{0,0,250}
\FXRegisterAuthor{ds}{sds}{\color{mycolor}DS}


\usepackage{pdfpages}
\setlength{\parindent}{1cm} %sangria

%%%%%%%%% COMANDOS PERSONALIZADOS
%% para notacion de conjuntos
\newcommand{\N}{\mathbb{N}}
\newcommand{\Z}{\mathbb{Z}}
\newcommand{\Q}{\mathbb{Q}}
\newcommand{\I}{\mathbb{I}}
\newcommand{\R}{\mathbb{R}}
\newcommand{\C}{\mathbb{C}} 
\newcommand{\F}{\mathbb{F}} 

%% FUNCIONES 
\newcommand{\f}{\textit{f}} 
\newcommand{\g}{\textit{g}}

\newcommand{\Pos}{\mathbb{P}} %Reales positivos
\newcommand{\Hilbert}{\mathcal{H}} % Espacio de Hilbert

\newcommand{\kernel}{\mathscr{N}}
\newcommand{\range}{\mathcal{R}} 
\newcommand{\lagran}{\mathcal{L}} 
\newcommand{\laplace}{\mathscr{L}} 
\newcommand{\partition}{\mathfrak{z}}
\newcommand{\M}{\mathcal{M}} %Matrices
\newcolumntype{E}{>{$}c<{$}} %entorno matematico en columnas de una tabla
\newcommand{\vi}{\boldsymbol{\hat{\imath}}}
\newcommand{\vj}{\boldsymbol{\hat{\jmath}}}
\newcommand{\vk}{\vu{k}}
\newcommand{\vr}{\hat{r}}
\newcommand{\vp}{\boldsymbol{\hat{\phi}}}
\newcommand{\vz}{\vu{z}}
\newcommand{\vaz}{\boldsymbol{\hat{\theta}}}
\newcommand{\vx}{\vu{x}}%vectores
\newcommand{\vy}{\vu{y}}%vectores 
\newcommand\numberthis{\addtocounter{equation}{1}\tag{\theequation}}
\newcommand{\LI}{\lim _{h\longrightarrow 0}}
\newcommand{\SU}{\longrightarrow \sum _{n=0} ^{\infty}}
\newcommand{\QED}{\hfill {\blackbox}}
\newcommand{\cis}{\text{cis} \,}
\newcommand{\sol}{\textbf{\underline{Solución}: }} %% Solucion
\newcommand{\af}{\textbf{\underline{Afirmación}: }}
\newcommand{\cej}{\textbf{\underline{Contraejemplo}: }}


%%%%%%%%%
%%%%%%% ENCABEZADO ^^^^^^^^^^^^^^^^^^^^$

\makeatletter
\let\thetitle\@title
\makeatother



\newcommand{\hoofding}[6]{
  \noindent\begin{minipage}{0.5\textwidth}
    \includegraphics[height=4cm]{img/ECFM-LOGO.jpeg}
  \end{minipage}%
  \begin{minipage}{0.5\textwidth}
    \raggedleft
    #1\\
    #2\\
    Nombre: #3\\
    Curso: #4\\
    Carne: #5\\
    Fecha: #6\\
  \end{minipage}

  \noindent\rule{\textwidth}{0.4pt} % Recta horizontal

  \vspace{5mm}
}
\newcommand{\figuur}[2]{\includegraphics[width=#1]{#2}} 



%hiermee is er geen insprong bij een nieuwe alinea
\setlength{\parindent}{0pt}
%de witruimte tussen twee alinea's wordt iets groter
\setlength{\parskip}{1ex plus 0.5ex minus 0.2ex}



%%%%%%% --------------------------------

%%ESPACIO DONDE SE LLLAMA EL ENCABEZADO

%\input{Header.tex}

\newtheoremstyle{Tema}% name of the style to be used
  {0mm}% measure of space to leave above the theorem. E.g.: 3pt
  {10mm}% measure of space to leave below the theorem. E.g.: 3pt
  {}% name of font to use in the body of the theorem
  {}% measure of space to indent
  {\bfseries}% name of head font
  {\newline}% punctuation between head and body
  {30mm}% space after theorem head
  {}% Manually specify head

%%%%%%%% FORMA DE HACER TEOREMAS - hacerlo manana
\newtheorem{theorem}{Teorema}
\newtheorem{example}{Ejemplo}
\newtheorem{corollary}{Corolario}
\newtheorem{lemma}{Lemma}
\newtheorem{definition}{Definicion}
\newtheorem{prop}{Proposicion}


\newtheorem{prob}{Problema}



\theoremstyle{Tema} \newtheorem{problem}{Problema}    %%%%%  Template para Problema



